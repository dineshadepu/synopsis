%============================= abs.tex================================
\begin{Abstract}
  In the present work, we develop a coupled formulation of Smoothed Particle
  Hydrodynamics (SPH) - Discrete Element Method (DEM) to handle fluid and solid
  mechanics problems. The framework is applied to model several different types
  of physics, such as fluid dynamics, elastic-plastic dynamics, rigid body
  dynamics, rigid-fluid coupling (RFC), fluid-structure interaction (FSI), and
  solid particle erosion (SPE). Applications such as abrasive water jet
  machining, fluid-bed reactors, and sediment transport are a few areas where
  these physical phenomena are dominant. We use SPH to model the fluid and
  elastic-plastic dynamics, while DEM handles the interaction among the
  arbitrarily shaped bodies. We implement the current work in the open-source
  framework PySPH. We use an automan package to automate all our results and
  make it fully reproducible. We produce an open-source framework of
  the current formulation in handling all these different physical processes.


  We first develop a technique in SPH to model fluid and elastic dynamics.
  Particle shifting techniques (PST) in SPH have been used to improve the
  accuracy of the SPH method. Shifting ensures that the particles are
  distributed homogeneously in space. This may be performed by moving the
  particles using a transport velocity. In this work, we propose an extension
  to the class of Transport Velocity Formulation (TVF) methods. We derive the
  equations in a consistent manner and show that there are additional terms that
  significantly improve the accuracy of the method. In particular, we apply this
  to the Entropically Damped Artificial Compressibility SPH method. We identify
  the free-surface particles and their normals using a simple approach and
  thereby adapt the method for free-surface problems. We show how the new method
  can be applied to the problem of elastic dynamics. We consider a suite of
  benchmark problems involving both fluid and solid mechanics to demonstrate the
  accuracy and applicability of the method.

  We extend the CTVF technique to handle the collision among the frictional
  elastic solids using a contact force model. Classically, smoothed particle
  hydrodynamics (SPH) is used to model collision between elastic solids, which
  involves handling the deformation of bodies and the interaction with other
  bodies. However, this does not allow one to incorporate friction and also
  subjects the solids to unphysical interaction when the bodies are nearby but
  not in actual contact. In the current work, we incorporate an efficient
  contact force model into a variant of the updated-Lagrangian SPH to model the
  interaction between elastic solids. This approach models friction as well as
  eliminates the spurious interaction between nearby bodies. The current model
  is validated with several numerical examples involving a collision between two
  and multiple elastic solids, with and without friction. These results compare
  well with finite element modeling (FEM), analytical, and experimental studies.
  We discuss the implementation details of contact force formulation to handle contact
  between the colliding bodies in PySPH. The model is easy to incorporate in any
  updated-Lagrangian SPH scheme.


  We further study the deformation of elastic structures under hydrodynamic
  loads with the CTVF model. CTVF is adopted to handle both fluid and structural
  dynamics, as it can eliminate inhomogeneous particle distribution in fluid
  flow and tensile instability in elastic structures without any additional
  terms. A ghost particle-based approach is used to handle the coupling between
  the fluid and solid phases. The proposed scheme is verified by simulating a
  few test cases, which are validated with exact analytical solutions. An
  elastic plate deformation due to dam-breaking fluid is simulated as an
  application. We outline sub-stepping algorithm used to update the states of
  two different materials for computationally efficient simulation.

  Next, we study the two-way coupling between the rigid body and the fluid flow
  using a particle-based framework. The interaction between the rigid bodies is
  handled with the discrete element method (DEM), where the contact force model
  developed to handle the collision between arbitrarily shaped elastic solids is
  used. The fluid phase is modeled with CTVF. The interaction between the fluid
  and rigid bodies is modeled using a dummy particle approach which is similar
  to handling the interaction between the fluid and the elastic structure. The
  accuracy of the developed method is evaluated via several numerical examples
  involving analytical and experimental results. The rigid-rigid interaction
  part of the solver is validated through the study of rolling and sliding body
  cases. The collapse of a stack of cylinders under gravity is studied and
  compared against its experimental counterpart. The rigid-fluid interaction is
  studied with the water entry of a rigid cube and with a rising cylindrical body
  case in the hydrostatic water tank.

  We finally model the erosion of a ductile solid due to the impact of multiple
  arbitrarily shaped projectiles. We model the elastic-plastic behavior of the
  ductile solid by incorporating a Johson-Cook constitutive model in CTVF. We
  adapt the contact force between the elastic bodies formulation to
  rigid-elastic solids. We further build a framework that can handle erosion of
  the target due to the impact of multiple bodies, which can interact or do not
  interact among themselves. 2D and 3D numerical cases are simulated to
  demonstrate the developed framework abilities. Erosion of a 2D ductile target
  due to multiple square particle impacts, with cases where square particles can
  self-interact and with no self-interaction, is simulated. A rotating spherical
  particle impacting a ductile target at different impact velocities is
  considered part of the 3D case.

%
%
%
%
%
\end{Abstract}
%=======================================================================
