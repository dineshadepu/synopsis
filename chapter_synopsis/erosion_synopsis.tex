\FloatBarrier%
\chapter{Erosion}
\label{chap:erosion}
We develop a numerical implementation to study the solid particle erosion of a
ductile target. We follow the numerical model proposed in
\citep{dong2016smoothed}. We utilize the CTVF to model the elastic behavior of
the ductile target, while the collision between the rigid bodies, elastic bodies
to modeled using DEM. We extend the developed numerical method to handle plastic
using a Johnson-Cook constitutive law. The objective of this work is to develop
an open-source framework to handle solid-particle erosion in two and three
dimensions. The solver should handle erosion of the target due to multiple
impacting particles. Here, The impacting particles may be interacting
themselves.


% ====================================================================================
% ====================================================================================
\FloatBarrier%
\section{Solid-particle erosion due to multiple square particles: with no self
  interaction}
\label{sec:res:mpe-2}

\Cref{fig:mpe-2-full} shows the snapshots for computational model 2. The target
and the body are discretized into $12708$ particles in the current case. From
\cref{fig:mpe-2-full,fig:mpe-2-border}, we can see that the both computational
models result in same erosion of the target, which is verified qualitatively by
comparing the eroded regions of of the target. For a quantitative validation, we
compare the y component of the center of mass of the bottom square particle with
time, when simulated using both the models in \cref{fig:mpe-2-ycom-vs-t}. From
\cref{fig:mpe-2-ycom-vs-t}, we can see that both the computational models give
the same result. Though both the models result in same erosion of the target,
however, computational model 1 is $2.3$ times faster than computational model 2.
This is due to boundary particle representation in the 1st model. A clearer
description is shown in \cref{table:mpe-2-time-comparison}.
\begin{figure}[!htpb]
  \centering
  \begin{subfigure}{0.48\textwidth}
    \centering
    \includegraphics[width=\textwidth]{figures/erosion/figures/multi_body_erosion_example_2/all_particles_no_self_intersection/all_bodies_time0.pdf}
    \subcaption{}
    \label{fig:mpe-2-full-a}
  \end{subfigure}
  \begin{subfigure}{0.48\textwidth}
    \centering
    \includegraphics[width=\textwidth]{figures/erosion/figures/multi_body_erosion_example_2/all_particles_no_self_intersection/all_bodies_time1.pdf}
    \subcaption{}
    \label{fig:mpe-2-full-b}
  \end{subfigure}
  \caption{Erosion of a ductile target due to the impact of two non-interacting
    square particles with computational model 2.}
\label{fig:mpe-2-full}
\end{figure}

% ====================================================================================
% ====================================================================================
\FloatBarrier%
\section{Solid-particle erosion due to multiple square particles: with
  self interaction}
\label{sec:erosion-multiple-impact-self-interact}

\Cref{fig:mpe-1-border} shows the snapshot of the target during the impact for
the computational model 1. \Cref{fig:mpe-1-border-a} shows the penetration of
body $1$ into the ductile target, and initial impact of the body $2$ with body
$1$. \Cref{fig:mpe-1-border-b} shows the snapshot after body $1$ loses the
contact with the target. \Cref{fig:mpe-4-border} shows the snapshot of the
target during the impact for the computational model 1 with a single square
particle. From \cref{fig:mpe-4-border}, we can see that the impacting body
erodes the target symmetrically and after losing contact it moves away from the
away without acquiring a rotation. However, for the two square particle impact
case, the material is eroded more from the right side, this is due to the second
square particle impact, this can be seen in \cref{fig:mpe-1-border}.

We repeat the same experiment with computational model 2. The erosion of the
ductile solid with computational model 2 is shown in \cref{fig:mpe-1-full}. From
\cref{fig:mpe-1-full,fig:mpe-1-border}, we can see that both the computational
models results in same erosion pattern. For quantitatively validation for the
similarity, we plot center of mass of the 1st body with time in
\cref{fig:mpe-1-ycom-vs-t}. From \cref{fig:mpe-1-ycom-vs-t}, we can see that
both 1 behaves same in both the computational models. The performance comparison
of the proposed computational models are shown in
\cref{table:mpe-1-time-comparison}.
\begin{figure}[!htpb]
  \centering
  \begin{subfigure}{0.48\textwidth}
    \centering
    \includegraphics[width=\textwidth]{figures/erosion/figures/multi_body_erosion_example_1/all_particles_self_intersection/all_bodies_time0.pdf}
    \subcaption{}
    \label{fig:mpe-1-full-a}
  \end{subfigure}
  \begin{subfigure}{0.48\textwidth}
    \centering
    \includegraphics[width=\textwidth]{figures/erosion/figures/multi_body_erosion_example_1/all_particles_self_intersection/all_bodies_time1.pdf}
    \subcaption{}
    \label{fig:mpe-1-full-b}
  \end{subfigure}
  \caption{Erosion of a ductile target due to the impact of two interacting
    square particles with computational model 2.}
\label{fig:mpe-1-full}
\end{figure}

\begin{table}[!htpb]
\centering
\begin{tabular}{c c c c c}
  \hline
  Model & No. particles & Time taken & Scale up  \\
  \hline
  Computational model 1 & $8476$ & $239.36$ sec & $1.956$ \\
  Computational model 2 & $12708$ & $468.21$ sec & 1 \\
\end{tabular}
\caption{CPU time comparison of computational model 1 and computational model 2
  for solid particle erosion for interacting particles.}
\label{table:mpe-1-time-comparison}
\end{table}
The demonstrated test cases show that the developed software can handle the
modeling erosion of a body in 2 and 3 dimensions. Further, we verified that the
representation of projectiles with border particles reduces the time of
computation. We established the current framework for handling erosion due to
multiple particle impacts.
