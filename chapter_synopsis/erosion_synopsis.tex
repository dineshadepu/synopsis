\FloatBarrier%
\chapter{Erosion}
\label{chap:erosion}
We develop a numerical implementation to study the solid particle erosion of a
ductile target. We follow the numerical model proposed in
\citep{dong2016smoothed}. We utilize the CTVF to model the elastic behavior of
the ductile target, while the collision between the rigid bodies, elastic bodies
to modeled using DEM. We extend the developed numerical method to handle plastic
using a Johnson-Cook constitutive law. The objective of this work is to develop
an open-source framework to handle solid-particle erosion in two and three
dimensions. The solver should handle erosion of the target due to multiple
impacting particles. Here, The impacting particles may be interacting
themselves.


% ====================================================================================
% ====================================================================================
\FloatBarrier%
\section{Solid-particle erosion due to multiple square particles: with no self
  interaction}
\label{sec:res:mpe-2}
We study the erosion of AL6061-T6 material due to the impact of multiple square
particles. We assume that the square particles do not interact with other square
particles. This test is used to demonstrate the capabilities of the developed
solver for handling the erosion of a target due to multiple non-interacting
particles.

\Cref{fig:mpe-2-full} shows the snapshots of the square particles, including the
target at two time instants. \Cref{fig:mpe-2-full-a} shows the square particle
and the target during the first impact. \Cref{fig:mpe-2-full-b} shows the second
particle impacting the target. From the figure, we can see the chip separation.
From \cref{fig:mpe-2-full-b} we can see the material eroded due to the first
particle impact flying away. Since, we assume square particles to be non interacting
we see the square particles to be overlapping.
\begin{figure}[!htpb]
  \centering
  \begin{subfigure}{0.48\textwidth}
    \centering
    \includegraphics[width=\textwidth]{figures/erosion/figures/multi_body_erosion_example_2/all_particles_no_self_intersection/all_bodies_time0.pdf}
    \subcaption{}
    \label{fig:mpe-2-full-a}
  \end{subfigure}
  \begin{subfigure}{0.48\textwidth}
    \centering
    \includegraphics[width=\textwidth]{figures/erosion/figures/multi_body_erosion_example_2/all_particles_no_self_intersection/all_bodies_time1.pdf}
    \subcaption{}
    \label{fig:mpe-2-full-b}
  \end{subfigure}
  \caption{Erosion of a ductile target due to the impact of two non-interacting
    square particles with computational model 2.}
\label{fig:mpe-2-full}
\end{figure}



% ====================================================================================
% ====================================================================================
\FloatBarrier%
\section{Solid-particle erosion due to multiple square particles: with
  self interaction}
\label{sec:erosion-multiple-impact-self-interact}
We study the erosion of AL6061-T6 material due to multiple square particle
impacts. In contrast to the previous section, we allow square particles to
interact among themselves.

\begin{figure}[!htpb]
  \centering
  \begin{subfigure}{0.48\textwidth}
    \centering
    \includegraphics[width=\textwidth]{figures/erosion/figures/multi_body_erosion_example_1/all_particles_self_intersection/all_bodies_time0.pdf}
    \subcaption{}
    \label{fig:mpe-1-full-a}
  \end{subfigure}
  \begin{subfigure}{0.48\textwidth}
    \centering
    \includegraphics[width=\textwidth]{figures/erosion/figures/multi_body_erosion_example_1/all_particles_self_intersection/all_bodies_time1.pdf}
    \subcaption{}
    \label{fig:mpe-1-full-b}
  \end{subfigure}
  \caption{Erosion of a ductile target due to the impact of two interacting
    square particles with computational model 2.}
\label{fig:mpe-1-full}
\end{figure}
\Cref{fig:mpe-1-full} shows the snapshot of the target during the square
particle impacts. \Cref{fig:mpe-1-full-a} shows the penetration of body $1$ into
the ductile target, and initial impact of the top body with bottom one.
\Cref{fig:mpe-1-full-b} shows the snapshot after both the bodies loses contact
with the target. from \cref{fig:mpe-1-full}, we can see that more material is
eroded from the right side, this is due to the upper square body impact
impacting the lower body.


We have demonstrated that the developed software can handle the modeling erosion
of a body in 2 and 3 dimensions. Further, we verified that the representation of
projectiles with border particles reduces the time of computation.
