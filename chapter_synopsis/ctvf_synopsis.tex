\FloatBarrier%
\chapter{Numerical modeling of fluid and elastic dynamics}
\label{chap:ctvf}

\begin{equation}
  \label{eq:ce}
  \frac{d \rho}{d t} = - \rho \; \frac{\partial u_i}{\partial x_i},
\end{equation}
and conservation of linear momentum,
\begin{equation}
  \label{eq:me}
  \frac{d u_i}{d t} = \frac{1}{\rho} \; \frac{\partial \sigma_{ij}}{\partial x_j}
  + g_i,
\end{equation}

where $\rho$ is the density, $u_i$ is the $i$\textsuperscript{th} component of
the velocity field, $x_j$ is the $j$\textsuperscript{th} component of the
position vector, $g_i$ is the component of body force per unit mass and
$\sigma_{ij}$ is stress tensor.


The stress tensor is split into isotropic and deviatoric parts,
\begin{equation}
  \label{eq:stress_tensor_decomposition}
  \sigma_{ij} = - p \; \delta_{ij} + \sigma'_{ij},
\end{equation}
%
where $p$ is the pressure, $\delta_{ij}$ is the Kronecker delta function, and
$\sigma'_{ij}$ is the deviatoric stress.

The Jaumann's formulation for Hooke's stress provides us with the rate of
change of deviatoric stress,
\begin{equation}
  \label{eq:jaumann-stress-rate}
  \frac{d \sigma'_{ij}}{dt} = 2G (\dot{\epsilon}_{ij} - \frac{1}{3}
  \dot{\epsilon}_{kk} \delta_{ij}) + \sigma^{'}_{ik}  \Omega_{jk} +
  \Omega_{ik} \sigma^{'}_{kj},
\end{equation}
where $G$ is the shear modulus, $\dot{\epsilon}_{ij}$ is the strain rate tensor,
\begin{equation}
  \label{eq:strain-tensor}
  \dot{\epsilon}_{ij} = \frac{1}{2} \bigg(\frac{\partial u_i}{\partial x_j} +
  \frac{\partial u_j}{\partial x_i} \bigg),
\end{equation}
and $\Omega_{ij}$ is the rotation tensor,
\begin{equation}
  \label{eq:rotational-tensor}
  \Omega_{ij} = \frac{1}{2} \bigg(\frac{\partial u_i}{\partial x_j} -
  \frac{\partial u_j}{\partial x_i} \bigg).
\end{equation}

For a weakly-compressible or incompressible fluid, a viscous force is added:
\begin{equation}
  \label{eq:fluid-stress-decomposition}
  \sigma_{ij} = - p \delta_{ij} + 2 \eta \frac{\partial u_i}{\partial x_j}
\end{equation}
where $\eta$ is the kinematic viscosity of the fluid.

In both fluid and solid modelling the pressure is computed using an
isothermal equation of state, given as,
\begin{equation}
  \label{eq:pressure-equation}
  p = K \bigg(\frac{\rho}{\rho_{0}} - 1 \bigg),
\end{equation}
where $K = \rho_{0} c_0^2$ is the bulk modulus. Here, the constants $c_0$ and
$\rho_0$ are the reference speed of sound and density, respectively. For solids,
$c_0$ is computed as $\sqrt{\frac{E}{3 (1 - 2 \nu)\rho_{0}}}$, $\nu$ is the
Poisson ratio, $E$ is the Young's modulus.

\section{Numerical method}
In CTVF we move the particles with a \emph{transport velocity},
$\tilde{\ten{u}}$. The material derivative in this case is written as,
\begin{equation}
  \label{eq:modified-material-derivative}
  \frac{\tilde{d} }{d t} = \frac{\partial }{\partial t} +
  \tilde{u}_j \frac{\partial }{\partial x_j}.
\end{equation}

We therefore recast the governing equations to incorporate the transport
velocity. The new governing equations are,
\begin{equation}
  \label{eq:ce-tvf}
  \frac{\tilde{d} \rho}{d t} =
  - \rho \frac{\partial \tilde{u}_j}{\partial x_j} +
  \frac{\partial (\rho (\tilde{u}_j - u_j))}{\partial x_j}.
\end{equation}
We therefore write the momentum equation as,
\begin{equation}
  \label{eq:mom-tvf}
  \frac{\tilde{d} u_i}{d t} =
  \frac{\partial}{\partial x_j} (u_i (\tilde{u}_j - u_j))
  - u_i \frac{\partial}{\partial x_j} (\tilde{u}_j - u_j)
  + g_i
  +\frac{1}{\rho} \frac{\partial \sigma_{ij}}{\partial x_j}.
\end{equation}
We note that this equation encompasses both fluid dynamics as well as elastic
dynamics by simply changing the way $\sigma_{ij}$ is modeled. The first term
on the right-hand-side of \cref{eq:mom-tvf} is the additional artificial
stress term that is included in the TVF~\citep{Adami2013}. The second term
involves the divergence of the transport velocity field. In the case of the
TVF, the term includes a background pressure acceleration that is of the form,
\begin{equation}
  \label{eq:tvf-accel}
  \bigg(\frac{d \ten{u}_a}{dt}\bigg)_{c} = - p^0_a \sum_{b \in N(a)}
  \frac{m_b}{\rho_b^2} \nabla W(\ten{r}_{ab}, \tilde{h}_{ab}),
\end{equation}
where $p^0_a$ is the background pressure for the given particle $a$,
$\ten{r}_{ab} = \ten{r}_a - \ten{r}_b$, $\tilde{h}_{ab} = (h_a + h_b)/2$, and
index $b$ refers to the neighbors of particle $a$. The additional terms arising
while adjusting the continuity and momentum to the transport velocity are
certainly not zero and therefore this should not be ignored. We investigate the
importance of including these terms in \cref{sec:results}. We note that in the
case of elastic dynamics that these terms are negligible and do not make a
significant difference. This has also been pointed out by
\textcite{zhang_hu_adams17}.

The Jaumann stress rate is also similarly modified to account for the
transport velocity as,
\begin{multline}
  \label{eq:modified-jaumann-stress-rate}
  \frac{\tilde{d} \sigma'_{ij}}{dt} = 2G (\dot{\epsilon}_{ij} - \frac{1}{3}
  \dot{\epsilon}_{kk} \delta_{ij}) + \sigma^{'}_{ik}  \Omega_{jk} +
  \Omega_{ik} \sigma^{'}_{kj} + \\
  \frac{\partial}{\partial x_k}\big(\sigma^{'}_{ij}  (\tilde{u}_k - u_k)\big)
  - \sigma^{'}_{ij} \frac{\partial}{\partial x_k} (\tilde{u}_k - u_k).
\end{multline}


The original EDAC-SPH ~\citep{edac-sph:cf:2019} scheme adjusted to
the transport velocity written as,
\begin{equation}
  \label{eq:edac-p-evolve}
  \frac{\tilde{d} p}{d t} =
  (p -\rho c_s^2)
    \text{div}(\ten{u})
  - p \; \text{div}(\tilde{\ten{u}})
    + \text{div}(p (\tilde{\ten{u}} - \ten{u}))
    + \nu_{edac}  \nabla^2 p.
\end{equation}
where $\nu_{edac}$ is a viscosity parameter for the smoothing of the pressure
and $c_s$ is the (artificial) speed of sound.
%
The value of $\nu_{edac}$ is,
\begin{equation}
  \label{eq:nu-edac}
  \nu_{edac} = \frac{\alpha_{\textrm{edac}} h c_s}{8},
\end{equation}
where $h$ is the smoothing length of the kernel and a value of
$\alpha_{\textrm{edac}}=0.5$ is recommended as suggested in~\citep{PRKP:edac-sph-iccm2015}.


\FloatBarrier%
\subsection{Lid driven cavity}
\label{sec:ldc}

We evaluate the ability of the proposed scheme to handle solid wall boundary
conditions by simulating a lid-driven cavity. The lid-driven cavity is a
classic problem that can be challenging to simulate in the context of the SPH.
It has been simulated by \citep{Adami2013}, \citep{huang_kernel_2019},
\citep{edac-sph:cf:2019} to note a few. A rectangular cavity with length 1 m
which is filled with fluid is constrained by four walls. Top wall has a
velocity of $U = 1 $ m\,s\textsuperscript{-1}. A unit density is assumed for the
fluid. The speed of sound of the fluid particle is set to $c = 10 U_{max}$. We
use the summation density to compute the density. The viscosity of the fluid
is set through the Reynolds number of the flow, $\nu = \frac{Re}{U}$. No
artificial viscosity is used in the current problem.

%
\begin{figure}
  \centering
  \includegraphics[width=0.5\linewidth]{figures/ctvf/figures/cavity/uv_re100}
  \caption{Velocity profiles $u$ vs.\ $y$ and $v$ vs.\ $x$ for the
    lid-driven-cavity problem at $Re=100$ with three initial particle
    arrangement of $50 \times 50$, $100 \times 100$, and $150 \times
    150$.}%
  \label{fig:ldc:uv_re100}
\end{figure}
%
\begin{figure}
  \centering
  \includegraphics[width=0.5\linewidth]{figures/ctvf/figures/cavity/uv_re1000}
  \caption{Velocity profiles for the lid-driven-cavity using the steady state
    simulation procedure for $Re = 1000$ with initial partial arrangement of
    $50 \times 50$, $100 \times 100$, and $200 \times 200$ compared with
    the results of~\citep{ldc:ghia-1982}.}%
\label{fig:ldc:uv_re1000}
\end{figure}

We now study convergence of the method as we vary the
resolution. \Cref{fig:ldc:uv_re100} and \cref{fig:ldc:uv_re1000} show the
center-line velocities $u$ versus $y$ and $v$ versus $x$ for the Reynolds
numbers 100 and 1000 respectively. For the $Re=100$ case we use three
different resolutions of $50\times 50, 100 \times 100$ and $150 \times
150$. For the $Re=1000$ case, we use an initial $50 \times 50$,
$100 \times 100$, and $200 \times 200$ grid of particles. These are compared
against the results of \citep{ldc:ghia-1982}. As we can see that the current
scheme is able to predict the velocity profiles well.


\FloatBarrier%
\subsection{Oscillating plate}
\label{sec:oscillating-plate}
We consider a thin oscillating plate that is clamped on one side.
An oscillating plate with a length of $0.2$ m and a height of $0.02$ m is
initially given with a velocity profile of,
%
\begin{equation*}
  v_y(x) = V_f \, c_0 \frac{F(x)}{F(L)},
\end{equation*}
where $V_f$ varies for different cases. $L$ is the length of the plate. $F(x)$
is given by,
\begin{multline}
  F(x) = (\cos(kL) + \cosh(kL)) \, (\cosh(kx) - \cos(kx)) + \\
  (\sin(kL) - \sinh(kL)) \, (\sinh(kx) - \sin(kx)).
\end{multline}
%
In the present example $kL$ is 1.875. The material properties of the plate are
as follows, Young's modulus $E=2.0\times 10^6$ Pa, a Poisson's ratio of
$\nu=0.3975$. $c_0$ is speed of sound, and a density of $\rho=1000$
kg\,m\textsuperscript{-3}, as done in~\citep{gray-ed-2001}.  In all the cases
simulated here, we use an $\alpha$ of $1$ for artificial viscosity.

This is further demonstrated by a case where an oscillating plate of length of
$0.2$ m and a height of $0.02$ m is simulated for a time of $0.22$ seconds.
Similarly, another case where plate of height $0.01$ m and a width of $0.2$ m is
run for a time of $0.51$ s.
\Cref{fig:oscillating-plate:etvf-sun2019-l-0-2-h-0-22} and
\cref{fig:oscillating-plate:etvf-sun2019-l-0-2-h-0-01} shows particles of the
plate at time $t=0.22$ s and $0.51$ s of these two cases respectively. As we can
see from the figure that the plate is free of numerical fracture, thus the
tensile instability is eliminated.
%
%
\begin{figure}
  \centering
  \includegraphics[width=0.8\columnwidth]{figures/ctvf/figures/oscillating_plate/etvf_sun2019_l_0_2_h_0_02}
  \caption{Oscillating plate at time $t=0.22$s with a length of $0.2$m and
    height of $0.02$m simulated with SPST with CTVF scheme.}
\label{fig:oscillating-plate:etvf-sun2019-l-0-2-h-0-22}
\end{figure}
%
%
\begin{figure}[!htp]
  \centering
  \includegraphics[width=0.8\columnwidth]{figures/ctvf/figures/oscillating_plate/etvf_sun2019_l_0_2_h_0_01}
  \caption{Oscillating plate at time $t=0.51$s with a length of $0.2$m and
    height of $0.01$m simulated with SPST with CTVF scheme.}
\label{fig:oscillating-plate:etvf-sun2019-l-0-2-h-0-01}
\end{figure}

The accuracy of the current scheme is evaluated by comparing with the
analytical results and with a convergence study. In
\cref{table:compare-analytical-with-simulated-h-l-0-1} we compare the time
period for the oscillation by the analytical and the numerical results with
varying $V_f$, where we consider an oscillating plate whose $H/L$ ratio is
$0.1$. The difference between the analytical result and the numerical result
is due to the fact that the analytical results are based on thin plate theory
where as the plate considered here has a finite thickness. Further, we can see
that the current numerical results are in agreement with the previously
reported numerical results~\citep{gray-ed-2001, zhang_hu_adams17}. In
\cref{fig:oscillating:ipst_convergence_plot}, we have performed a convergence
study of an oscillating plate, with a $\nu=0.3975$, and $V_f=0.05$
m\,s\textsuperscript{-1}, and IPST is used for particle homogenization. The trend of
the current scheme matches well with the other updated Lagrangian SPH
schemes~\citep{gray-ed-2001, zhang_hu_adams17}. Hence the current scheme is
able to work with different PST methods and removes the tensile instability.

\begin{table}[!htpb]
\centering
\begin{tabular}{c c c c c}
  \hline
  $V_f$ & 0.001 & 0.01 & 0.03 & 0.05 \\
  \hline
  $\text{T}_{\mathrm{CTVF}}$ & 0.284 & 0.283 & 0.283 & 0.284 \\
  $\text{T}_{\mathrm{GTVF}}$ & 0.284 & 0.283 & 0.284 & 0.285 \\
  $\text{T}_{\mathrm{analytical}}$ & 0.254 & 0.252 & 0.254 & 0.254
\end{tabular}
\caption{Comparison between the CTVF and the analytical solution for the time
  period of the oscillating plate with a length of $0.2$m and height of
  $0.02$m with various $V_f$}
\label{table:compare-analytical-with-simulated-h-l-0-1}
\end{table}
