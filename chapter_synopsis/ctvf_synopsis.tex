\chapter{Numerical modeling of fluid and elastic dynamics}
\label{chap:ctvf}

In the current work, to model the fluid and elastic-plastic dynamics, we have
developed a corrected transport velocity transport (CTVF) scheme. The scheme
builds on the original transport velocity formulation (TVF) scheme of
\textcite{Adami2013} and is an improvement on the generalized TVF (GTVF) of
\textcite{zhang2017generalized}. For smoother pressure distributions, we adapted
EDAC-SPH~\parencite{edac-sph:cf:2019} to the transport velocity formulation. We
consider the additional terms that are missing in \parencite{Adami2013} while
adjusting the continuity and momentum equation to the transport velocity. We
additionally explore two particle shifting techniques (PST) to transport the
particles.

We demonstrate the importance of additional terms through a lid driven cavity
example. We consider a rectangular cavity with length 1 m which is filled with
fluid is constrained by four walls. Top wall has a velocity of $U = 1 $
m\,s\textsuperscript{-1}. The viscosity of the fluid is set through the Reynolds
number of the flow, $\nu = \frac{Re}{U}$.
%
\begin{figure}
  \centering
  \includegraphics[width=1.0\linewidth]{figures/ctvf/figures/cavity_sun2019_corrections_test/good_vs_bad}
  \caption{ Particle plot of cavity with a $Re=100$ with particle arrangement
    of $150 \times 150$, left side with corrections and right side without
    correction terms.}%
  \label{fig:ldc:particle_plots_re100_compare}
\end{figure}
%
We first simulate the cavity problem with a Reynolds number of 100 with and
without corrections. In \cref{fig:ldc:particle_plots_re100_compare} we can see
that an unphysical void is produced when no corrections are employed. This is
eliminated with the current scheme.


We next consider an oscillating plate with CTVF. Initially the plate is with a velocity
profile of,
%
\begin{equation*}
  v_y(x) = V_f \, c_0 \frac{F(x)}{F(L)},
\end{equation*}
where $V_f$ varies for different cases. $L$ is the length of the plate. $F(x)$
is given by,
\begin{multline}
  F(x) = (\cos(kL) + \cosh(kL)) \, (\cosh(kx) - \cos(kx)) + \\
  (\sin(kL) - \sinh(kL)) \, (\sinh(kx) - \sin(kx)).
\end{multline}
%
The material properties of the plate are as follows, Young's modulus
$E=2.0\times 10^6$ Pa, a Poisson's ratio of $\nu=0.3975$. $c_0$ is speed of
sound, and a density of $\rho=1000$ kg\,m\textsuperscript{-3}. We can see from
the \cref{fig:oscillating-plate}, that the plate is free of numerical fracture.
Thus we can say that the current solver is able to eliminate tensile instability
with the transport velocity formulation and is able to simulate both fluids and
solids.
\begin{figure}[tpb]
  \centering
  \includegraphics[width=0.4\textwidth]{figures/ctvf/figures/oscillating_plate_large_deformation/ctvf_ipst}
  \caption{Oscillating plate at time $t=0.359$ seconds with a length of $0.4$m and
    height of $0.02$m simulated with IPST with CTVF scheme.}
\label{fig:oscillating-plate}
\end{figure}
