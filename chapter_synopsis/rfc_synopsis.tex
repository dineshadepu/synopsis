\FloatBarrier%
\chapter{RFC}
\label{chap:rfc}
We model the dynamics of rigid bodies in fluid flow and the coupled behavior of
fluid and rigid bodies. Transport of arbitrarily shaped rigid bodies in fluid
flows is a common phenomenon that occurs widely in nature. We couple CTVF with
DEM to handle the rigid fluid coupling problems. The fluid phase is modeled
using a corrected transport velocity formulation. CTVF provides smooth pressure
distribution with EDAC formulation and homogeneous particle distribution,
resulting in accurate fluid modeling. Rigid-rigid interactions are modeled with
DEM. The interaction between the fluid phase and rigid bodies is handled using
the dummy particle approach.

We demonstrate the CTVF-DEM model by simulating sliding of a rigid cube on a
frictional inclined plane, and rising of a cylinder in a
hydrostatic tank.


\FloatBarrier%
\section{Rigid body sliding down an inclined plane}
\label{sec:rigid-body-sliding}
The rigid body of length $0.1$ m, height
$0.1$ m, is allowed to slide on a frictional surface which is at an angle
$\frac{\pi}{3}$. The schematic is shown is \cref{fig:rigid_body_sliding}. A
density of $2000$ kg\,m\textsuperscript{-3} is used for the body.
\begin{figure}[!htpb]
  \centering
  \includegraphics[width=0.4\textwidth]{images/rfc/images/rigid_body_sliding/schematic}
  \caption{Schematic of a square body sliding down an inclined plane under gravity.}
\label{fig:rigid_body_sliding}
\end{figure}
We have considered three different friction coefficients, $\mu=0.2$, $0.3$, and
$0.6$. From the analytical solution, when the friction coefficient is greater
than $\tan(\frac{\pi}{3})$, we have no slip condition and the body doesn't
slide.

\Cref{fig:results-solid-sliding-velocity-vs-time-2d} shows a evolution of
velocity of the center of mass of the rigid body for different frictional
coefficients against the analytical solution. From
\cref{fig:results-solid-sliding-velocity-vs-time-2d} we can see that the current
solver has an excellent match with the analytical solution and covers all the
regimes of the sliding case.
\begin{figure}[!htpb]
  \centering
  \includegraphics[width=0.6\textwidth]{figures/rfc/figures/mohseni_2021_free_sliding_on_a_slope_2d/velocity_vs_time}
  \caption{Variation of the velocity of the rigid body with time for different
    friction coefficients. Present result is compared against the analytical
    result.}
\label{fig:results-solid-sliding-velocity-vs-time-2d}
\end{figure}


\FloatBarrier%
\section{Cylinder rising in a hydrostatic tank}
\label{sec:water-entry-sphere}
% https://www.sciencedirect.com/science/article/pii/S0997754621001412#fig2
We study the behavior of a circular cylinder of density $500$
kg\,m\textsuperscript{-3} immersed in a hydrostatic tank under gravity. Since
the density of the solid body is half of the fluid, the body will be half afloat
while coming to rest.
\begin{figure}[!htpb]
  \centering
    \includegraphics[width=0.4\textwidth]{images/rfc/images/water_entry_of_sphere/schematic}
  \caption{The schematic of an immersed cylinder in a hydrostatic tank.}
\label{fig:raising-falling-solid-in-water}
\end{figure}

\begin{figure}[!htpb]
  \centering
  \begin{subfigure}{0.48\textwidth}
    \centering
    \includegraphics[width=1.0\textwidth]{figures/rfc/figures/dinesh_2022_body_in_hs_tank_2d/time11}
  \end{subfigure}
  %
  \begin{subfigure}{0.48\textwidth}
    \centering
    \includegraphics[width=1.0\textwidth]{figures/rfc/figures/dinesh_2022_body_in_hs_tank_2d/ycom}
  \end{subfigure}
  \caption{
    On the left, a snapshot at $t=15$ s is shown. On the right,
    time variation of the center of mass is shown.}
\label{fig:snapshots-rising-solid-in-water}
\end{figure}
\Cref{fig:snapshots-rising-solid-in-water} presents snapshot of a rising
cylinder at time $t=15$ s and time variation of the center of mass with time. As
can be seen from the figure, since the density of the cylinders is 500
kg\,m\textsuperscript{-3} the cylinder floats at half of its diameter.
