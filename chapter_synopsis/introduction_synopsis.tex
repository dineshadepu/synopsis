\chapter{Introduction}
\label{chap:intro}
Several real-life applications involve fluid dynamics, structural dynamics, and
interaction between these two materials. Examples include fluid-structure
interaction, rigid body collision inside fluid flow, rigid bodies colliding,
\begin{figure}[!htpb]
  \centering
  \includegraphics[width=0.7\textwidth]{images/intro/images/intro_section/erosion_hand_drawn_2}
  \caption{A schematic of erosion of a ductile solid due to several arbitrarily
    shaped solids impacting it in a fluid flow.}
\label{fig:intro-big-picture}
\end{figure}
deforming, and eroding an elastic-plastic target. \Cref{fig:intro-big-picture}
depicts a complex physical process. From \Cref{fig:intro-big-picture}, we can
see the erosion of a ductile solid due to the impact of rigid bodies in a fluid
flow. From the numbered labels, one needs to model the following physics to
model a problem such as this,
\begin{enumerate}
\item fluid dynamics,
\item elastic-plastic dynamics,
\item fluid-structure interaction (FSI),
\item free rigid body dynamics,
\item rigid-fluid coupling (RFC) and rigid-rigid interaction and,
\item solid particle erosion (SPE).
\end{enumerate}
We employ particle-based numerical methods to model the physical processes
involved in handling a problems like in \cref{fig:intro-big-picture}, due to the
natural ability to handle free surfaces and large deformation problems. We use a coupled
smoothed particle hydrodynamics (SPH) and discrete element method (DEM)
formulation to model the physical processes. We now review the state of the art
in SPH and DEM to model these physical processes.


To model fluid and elastic dynamics, we choose SPH. SPH method has been
extensively applied to fluids, elastic dynamics, fluid-structure interaction
among other areas~\parencite{monaghan2012smoothed}. Due to its meshless and
Lagrangian nature the particles move with the local velocity and can introduce
disorder in the particles and thereby significantly reduce the accuracy of the
method. To ensures particle homogenization, \textcite{Adami2013} proposed
Transport Velocity Formulation (TVF) in SPH and applied to internal flow
problems, producing very accurate results. A generalized TVF by
\textcite{zhang_hu_adams17} is proposed to simulate free surface flow problems.
However, \textcite{Adami2013,zhang_hu_adams17} do not consider additional terms
while recasting the governing equations to the transport velocity formulation.
Further, GTVF leads to tensile instability in solids, when used with a different
PST.


In modeling the collision of elastic solids, SPH has been successful
\parencite{gray2001sph}. However, it does not consider friction between the
colliding solids. Another problem is that the model generates spurious forces on
bodies which are moving close to each other. \textcite{yan2021simulation}
introduced the interfacial SPH scheme, which eliminates the spurious
interactions but it cannot handle friction between the solids. Additionally the
interaction force does not consider the shape of the solids in contact.


FSI is modeled by several works using SPH, such as, weakly compressible
SPH-Total Lagrangian SPH (TLSPH), WCSPH-Updated Lagrangian SPH (ULSPH),
incompressible SPH-TLSPH~\parencite{khayyer2022systematic}. However, no work is
reported in an updated Lagrangian framework to model FSI with a transport
velocity formulation with corrections included.


In order to simulate rigid-rigid interaction between the bodies with irregular
geometries multi-sphere approach \parencite{kruggel-emden_study_2008}, and
surface mesh represented (SMR)-DEM ~\parencite{zhan2021surface} are used.
However, the multi-sphere approach fails to handle the contact accurately with
bodies involving sharp corners, as the force law assumes the contact between two
spherical particles. SMR-DEM requires additional information to handle the
collision, such as connectivity between the particles comprising the body in
addition to the particle positions. Additionally, we need additional sets of
particles to handle the interaction between the rigid body and the fluid
particles to model rigid-fluid coupling.


To simulate solid particle erosion, \textcite{dong2016smoothed} uses smoothed
particle hydrodynamics (SPH). However, \textcite{dong2016smoothed} does not
consider the arbitrary shape of the projectile. Further, collision among the
projectiles while interacting with the target is not modeled. To the authors
knowledge there is no open-source implementation in the literature which can
model the solid particle erosion using SPH.


Having discussed the drawbacks in modeling of these complex physical processes,
in the current work we develop an open-source framework to model the complex
physics. The following are the key goals of the work.
\begin{itemize}
\item Develop a unified technique in SPH to solve both fluid and solid dynamics
  problems.
\item Handle the collision between the elastic solids using a penalty-based contact force.
\item Handle the collision among the arbitrarily shaped rigid projectiles in fluid
  flow. Solve the two-way coupling between the fluid and the rigid bodies.
\item Develop a fluid-structure interaction solver.
\item Provide an open-source implementation for solid particle erosion in SPH.
\end{itemize}

We propose a corrected transport velocity formulation (CTVF) which works for both
solid mechanics and fluid mechanics problems. We study the importance of the
missing terms while recasting the governing equation to transport velocity
formulation. We employ the EDAC formulation to the transport velocity
formulation and show that there are additional correction terms in the EDAC
scheme that should be introduced to improve the accuracy of the method. The
method is also robust to the choice of the smoothing kernel.

We model the collision between elastic solids using a penalty-based contact
force model. The penalty-based force considered here is the one proposed by
\textcite{mohseni2021particle}. We explore the importance of choosing the
primary and secondary body under collision. We also eliminate the spurious
interaction forces between nearby interacting bodies and show how the contact
force can be used for elastic bodies. Additionally, friction between the
colliding solids is considered with the current formulation.

We model FSI problems with the CTVF method. We model both fluids and solid
material using CTVF alone. We couple CTVF with DEM to model the rigid fluid
coupling problems. The fluid phase is handled using the CTVF, which provides
smooth pressure distribution with EDAC formulation. DEM is used to handle
rigid-rigid interactions The interaction between the fluid phase and rigid
bodies is handled using the dummy particle approach \parencite{Adami2012}.

We develop a framework to model the solid particle erosion with CTVF method to
model the elastic-plastic behavior of the ductile target. The plastic
behavior is incorporated using a Johson-Cook constitutive model. The interaction
between the arbitrarily shaped solids and the ductile target are modeled with a
modified contact force formulation. We establish the framework such that it can
handle erosion due to multiple impacts. Here, we allow the multiple particles to
also interact among themselves.



In the next few sections we give an overview of the our work in modeling these
physical processes.
