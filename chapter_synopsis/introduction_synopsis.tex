\chapter{Synopsis}
\label{chap:synopsis}



\section{Introduction}
\label{sec:intro}
In the current chapter we model the collision between elastic bodies. Modeling
the collision among elastic solids allows us to handle the interaction between
the target and the impactor when they are assumed as elastic or elastic-plastic.
An SPH based solution leads to unphysical interactions in modeling the collision
of objects when they are close by but not touching. Further, SPH model does not
incorporate friction between the colliding bodies. In the current chapter, a
contact force model is incorporated into the existing solver to handle the
interaction between the colliding solids. Specifically, we consider bodies with
arbitrary shapes.


The collision of arbitrary solid bodies occurs everywhere around us. Apart
from the elastic collision of bodies, other important examples include surface
erosion, waterjet machining \citep{natarajan2020abrasive}, and machining
processes \citep{islam2020numerical} to note a few. It is important to be able
to simulate such problems accurately.

% When body A discretized into n particles is colliding a body B which is
% discretized in to m particles, above techniques use the particles in B to
% compute the accelerations of density and velocity of particles in A. A few
% drawbacks of this approach is, the bodies come in to contact even when the
% bodies are physically not touching, and the no frictional modeling is modeled.
% \cite{yan2021simulation} has proposed a interfacial SPH scheme , where each
% elastic body uses their own particles in computation of divergence, momentum
% and strain rate. While, the bodies interact by a repulsive force inspired from
% Ansys. Where they have shown the improvement on modeling the collision with
% different examples. Yan has considered Gray SPH which is not so great with
% kernel invariance as it shows tensile instability with a few kernels as shown
% by Roth 2020. Also the contact force model didn't incorporate the frictional
% part of the collision. Further the contact force has a influence of 2h radius,
% implies there would be a force acting on the interacting bodies though they
% are not touching physically.

% The contact force model used in Yan is inspired from ANSYS model and is given.
% A similar work is done by \cite{vyas2021collisional}, where the author models
% the interaction between a circular rigid body impacting a target surface. In
% his work, author proposes a non-linear penalty based force model which could
% handle friction as well and studies the rebounding characteristic of the
% impactor. Here, the author studies the post collision behaviour of the
% circular rigid body which is interacting the elastic plastic solid at
% different incident angles. \cite{mohseni2021particle} proposed an advanced
% version of the collision model where he models the interaction between a rigid
% body with a brittle solids and models the erosion of the brittle solid.

% Finally the current model comes into place only when the material interfaces
% are touching unlike what Yan's contact force model and conventional SPH. We
% also check the invariance of the current scheme in modeling the collisions for
% different types of kernels used.


In the current work, the collision between elastic solids is modeled using a
penalty-based contact force model. A contact force based approach for collision
handling will eliminate the spurious interaction between bodies, which occurs
while modeling with an SPH-based model. Unlike the approach of
\citep{yan2021simulation}, the proposed contact force model can handle friction
between the solids as well. The bodies themselves are elastic and this is
simulated using the CTVF SPH method \citep{adepu2021corrected}. The
penalty-based force considered here is the one proposed by
\cite{mohseni2021particle}. In the original model proposed by
\citep{mohseni2021particle}, the contact force is between a primary body and a
secondary body. In \citep{mohseni2021particle}, the primary body is usually
treated as the rigid body and the body on which the erosion is simulated is
treated as secondary. It is not clear what would happen if both bodies were
elastic or if there is no clear way to distinguish between a primary and
secondary body. We explore these questions in the present work. The work of
\cite{mohseni2021particle} takes inspiration from that of
\cite{vyas2021collisional} where too there is a clear distinction between
primary and secondary bodies. \cite{vyas2021collisional} also consider the
collision between a rigid and elastic body. In the present work we only
implement the collision model of \citep{mohseni2021particle} due to its
simplicity. The model proposed by \cite{vyas2021collisional} is more complex to
implement. Several examples are simulated to validate the current scheme,
ranging from simulations compared with FEM, analytical results as well as
experimental. We next look at the numerical method used to model the collision
among the elastic solids.
